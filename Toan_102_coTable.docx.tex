%!TEX xelatex
\documentclass{book}
\usepackage{moodle}
\usepackage{graphicx}
\usepackage{amssymb}
\usepackage{amsmath}
\begin{document}


\begin{quiz}{THPT 2020 – Môn Toán – Mã đề 102}

\begin{multi}[points=1]{Câu 1}
 Nghiệm của phương trình ${{\log }_{2}}\left( x+9 \right)=5$ là

\item  $x=41$.	
\item*  $x=23$.	
\item  $x=1$.	
\item  $x=16$.
\end{multi}

\begin{multi}[points=1]{Câu 2}
 Tập xác định của hàm số $y={{5}^{x}}$ là

\item*  $\mathbb{R}$.	
\item  $\left( 0;+\infty  \right)$.	
\item  $\mathbb{R}\backslash \left\{ 0 \right\}$.	
\item  $\left[ 0\,;\,+\infty  \right)$.
\end{multi}

\begin{multi}[points=1]{Câu 3}
 Với $a$ là số thực dương tùy ý, ${{\log }_{5}}\left( 5a \right)$ bằng

\item  $5+{{\log }_{5}}a$.	
\item  $5-{{\log }_{5}}a$.	
\item*  $1+{{\log }_{5}}a$.	
\item  $1-{{\log }_{5}}a$.
\end{multi}

\begin{multi}[points=1]{Câu 4}
 Đồ thị của hàm số nào dưới đây có dạng như đường cong trong hình bên? 

\item  $y=-{{x}^{4}}+2{{x}^{2}}-1$.	
\item  $y={{x}^{4}}-2{{x}^{2}}-1$.	
\item  $y={{x}^{3}}-3{{x}^{2}}-1$.	
\item*  $y=-{{x}^{3}}+3{{x}^{2}}-1$.
\end{multi}

\begin{multi}[points=1]{Câu 5}
 Trong không gian \[\text{Ox}yz\], cho đường thẳng $d:\frac{x-4}{2}=\frac{z-2}{-5}=\frac{z+1}{1}$. Điểm nào sau đây thuộc $d$?

\item*  $N(4;2;-1)$.	
\item  $Q(2;5;1)$.	
\item  $M(4;2;1)$.	
\item  $P(2;-5;1)$.
\end{multi}

\begin{multi}[points=1]{Câu 6}
 Trong không gian \[\text{Ox}yz\], cho mặt cầu$(S):{{(x+1)}^{2}}+{{(y+2)}^{2}}+{{(z-3)}^{2}}=9$. Tâm của $(S)$ có tọa độ là: 

\item  $(-2;-4;6)$.	
\item  $(2;4;-6)$.	
\item*  $(-1;-2;3)$.	
\item  $(1;2;-3)$.
\end{multi}

\begin{multi}[points=1]{Câu 7}
 Cho khối chóp có diện tích đáy $B=6{{a}^{2}}$ và chiều cao $h=2a$. Thể tích khối chóp đã cho bằng: 

\item  $2{{a}^{3}}$.	
\item*  $4{{a}^{3}}$.	
\item  $6{{a}^{3}}$.	
\item  $12{{a}^{3}}$.
\end{multi}

\begin{multi}[points=1]{Câu 8}
 Cho khối trụ có bán kính đáy bằng $r=5$ và chiều cao $h=3$. Thể tích của khối trụ đã cho bằng

\item  $5\pi $.	
\item  $30\pi $.	
\item  $25\pi $.	
\item*  $75\pi $.
\end{multi}

\begin{multi}[points=1]{Câu 9}
 Trên mặt phẳng tọa độ, điểm nào dưới đây là điểm biểu diễn số phức $z=1-2i$?

\item  $Q\left( 1\,;\,2 \right)$.	
\item  $M\left( 2\,;\,1 \right)$.	
\item  $P\left( -2\,;\,1 \right)$.	
\item*  $N\left( 1\,;\,-2 \right)$.
\end{multi}

\begin{multi}[points=1]{Câu 10}
 Cho hai số phức \[{{z}_{1}}=1+2i\] và \[{{z}_{2}}=4-i\]. Số phức \[{{z}_{1}}-{{z}_{2}}\] bằng

\item  \[3+3i\].	
\item  \[-3-3i\].	
\item*  \[-3+3i\].	
\item  \[3-3i\].
\end{multi}

\begin{multi}[points=1]{Câu 11}
 Cho mặt cầu có bán kính $r=5$. Diện tích mặt cầu đã cho bằng

\item  $25\pi $.	
\item  $\frac{500\pi }{3}$.	
\item*  $100\pi $.	
\item  $\frac{100\pi }{3}$.
\end{multi}

\begin{multi}[points=1]{Câu 12}
 Tiệm cận đứng của đồ thị hàm số $y=\frac{x-1}{x-3}$ là

\item  $x=-3$.	
\item  $x=-1$.	
\item  $x=1$.	
\item*  $x=3$.
\end{multi}

\begin{multi}[points=1]{Câu 13}
 Cho hình nón có bán kính đáy \[r=2\] và độ dài đường sinh $l=7$. Diện tích xung quanh của hình nón đã cho bằng

\item  $28\pi $.	
\item*  $14\pi $.	
\item  $\frac{14\pi }{3}$.	
\item  $\frac{98\pi }{3}$.
\end{multi}

\begin{multi}[points=1]{Câu 14}
 $\int{6{{x}^{5}}dx}$bằng

\item  $6{{x}^{6}}+C$.	
\item*  ${{x}^{6}}+C$.	
\item  $\frac{1}{6}{{x}^{6}}+C$.	
\item  $30{{x}^{4}}+C$.
\end{multi}

\begin{multi}[points=1]{Câu 15}
 Trong không gian $Oxyz$, cho mặt phẳng $\left( \alpha  \right):2x-3y+4z-1=0$. Vectơ nào dưới đây là một vectơ pháp tuyến của $\left( \alpha  \right)$? 

\item*  $\overrightarrow{{{n}_{3}}}=\left( 2;\,\,-3;\,\,4 \right)$.	
\item  $\overrightarrow{{{n}_{2}}}=\left( 2;\,\,3;\,\,-4 \right)$.	
\item  $\overrightarrow{{{n}_{1}}}=\left( 2;\,\,3;\,\,4 \right)$.	
\item  $\overrightarrow{{{n}_{4}}}=\left( -2;\,\,3;\,\,4 \right)$.
\end{multi}

\begin{multi}[points=1]{Câu 16}
 Cho cấp số cộng $\left( {{u}_{n}} \right)$ với ${{u}_{1}}=9$ và công sai $d=2$. Giá trị của ${{u}_{2}}$ bằng

\item*  $11$.	
\item  $\frac{9}{2}$.	
\item  $18$.	
\item  $7$.
\end{multi}

\begin{multi}[points=1]{Câu 17}
 Cho hàm số bậc bốn \[y=f(x)\] có đồ thị là đường cong trong hình vẽ bên. Số nghiệm thực của phương trình \[f(x)=-\frac{3}{2}\] là

\item*  \[4\]	
\item  \[1\]	
\item  \[3\]	
\item  \[2\]
\end{multi}

\begin{multi}[points=1]{Câu 18}
 Phần thực của số phức \[z=3-4i\] bằng

\item*  \[3\]	
\item  \[4\]	
\item  \[-3\]	
\item  \[-4\]
\end{multi}

\begin{multi}[points=1]{Câu 19}
 Cho khối lăng trụ có diện tích đáy $B=3$ và chiều cao $h=2$. Thể tích của khối lăng trụ đã cho bằng

\item  $1$.	
\item  $3$.	
\item  $2$.	
\item*  $6$.
\end{multi}

\begin{multi}[points=1]{Câu 20}
 Cho hàm số $f\left( x \right)$ có bảng biến thiên như sau: 
Điểm cực đại của hàm số đã cho là

\item  $x=3$.	
\item  $x=-1$.	
\item*  $x=1$.	
\item  $x=-2$.
\end{multi}

\begin{multi}[points=1]{Câu 23}
 Cho hàm số $y=f\left( x \right)$ có đồ thị là đường cong trong hình bên. Hàm số đã cho nghịch biến trên khoảng nào dưới đây? 

\item*  $\left( -1;\,0 \right).$	
\item  $\left( -\infty ;\,-1 \right)$.	
\item  $\left( 0;\,1 \right)$.	
\item  $\left( 0;\,+\infty  \right)$.
\end{multi}

\begin{multi}[points=1]{Câu 24}
 Nghiệm của phương trình ${{2}^{2x-4}}={{2}^{x}}$ là

\item  $x=16$.	
\item  $x=-16$.	
\item  $x=-4$.	
\item*  $x=4$.
\end{multi}

\begin{multi}[points=1]{Câu 25}
 Trong không gian $Oxyz$, điểm nào dưới đây là hình chiếu vuông góc của điểm $A\left( 1;2;3 \right)$ trên mặt phẳng $Oxy$.

\item  $Q\left( 1;0;3 \right)$	
\item*  $P\left( 1;2;0 \right)$	
\item  $M\left( 0;0;3 \right)$	
\item  $N\left( 0;2;3 \right)$
\end{multi}

\begin{multi}[points=1]{Câu 26}
 Cho hàm số $f\left( x \right)$ có đạo hàm x$f'\left( x \right)=x\left( x-1 \right){{\left( x+4 \right)}^{3}},\,\forall x\in \mathbb{R}$. Số điểm cực tiểu của hàm số đã cho là

\item*  	$23$
\item  	$34$
\item  	$43$
\item  $33s$
\end{multi}

\begin{multi}[points=1]{Câu 27}
 Với $a,\,\,b$ là các số thực dương tùy ý thỏa mãn ${{\log }_{3}}a-2{{\log }_{9}}b=2$, mệnh đề nào dưới đây đúng? 

\item  $a=9{{b}^{2}}$.	
\item*  $a=9b$.	
\item  $a=6b$.	
\item  $a=9{{b}^{2}}$.
\end{multi}

\begin{multi}[points=1]{Câu 28}
 Cho hình hộp chữ nhật $ABC\text{D}.A'B'C'D'$ có $AB=a$, $A\text{D}=2\sqrt{2}a$, \[AA'=\sqrt{3}a\] (tham khảo hình bên). Góc giữa đường thẳng $A'C$ và mặt phẳng $\left( ABC\text{D} \right)$ bằng

\item*  ${{45}^{{}^\circ }}$.	
\item  ${{90}^{{}^\circ }}$.	
\item  ${{60}^{{}^\circ }}$.	
\item  ${{30}^{{}^\circ }}$.
\end{multi}

\begin{multi}[points=1]{Câu 29}
 Cắt hình trụ $\left( T \right)$ bởi một mặt phẳng qua trục của nó, ta được thiết diện là một hình vuông cạnh bằng $1$. Diện tích xung quanh của $\left( T \right)$ bằng. 

\item*  $\pi $.	
\item  $\frac{\pi }{2}$.	
\item  $2\pi $.	
\item  $\frac{\pi }{4}$.
\end{multi}

\begin{multi}[points=1]{Câu 30}
 Trong không gian $Oxyz$, cho điểm $M\left( 2;1;-2 \right)$ và mặt phẳng $\left( P \right):3x-2y+z+1=0$. Phương trình của mặt phẳng đi qua $M$ và song song với $\left( P \right)$ là: 

\item  $2x+y-2x+9=0$.		
\item  $2x+y-2z-9=0$	
\item  $3x-2y+z+2=0$.		
\item*  $3x-2y+z-2=0$.
\end{multi}

\begin{multi}[points=1]{Câu 31}
 Gọi ${{z}_{1}}$ và ${{z}_{2}}$ là hai nghiệm phức của phương trình \[{{z}^{2}}-z+3=0\]. Khi đó \[\left| {{z}_{1}} \right|+\left| {{z}_{2}} \right|\] bằng

\item  \[\sqrt{3}\].	
\item*  \[2\sqrt{3}\].	
\item  \[6\].	
\item  \[3\].
\end{multi}

\begin{multi}[points=1]{Câu 32}
 Giá trị nhỏ nhất của hàm số $f\left( x \right)={{x}^{4}}-12{{x}^{2}}-4$ trên đoạn $\left[ 0;9 \right]$ bằng

\item  \[-39\].	
\item*  \[-40\].	
\item  \[-36\].	
\item  \[-4\].
\end{multi}

\begin{multi}[points=1]{Câu 33}
 Cho số phức $z=2-i$, số phức $\left( 2-3i \right)\bar{z}$ bằng

\item  $-1+8i$.	
\item  $-7+4i$.	
\item*  $7-4i$.	
\item  $1+8i$.
\end{multi}

\begin{multi}[points=1]{Câu 34}
 Gọi $D$ là hình phẳng giới hạn bởi các đường $y={{e}^{4x}},y=0,x=0$ và $x=1$. Thể tích của khối tròn xoay tạo thành khi quay $D$ quanh trục $Ox$ bằng

\item  $\int\limits_{0}^{1}{{{e}^{4x}}\text{d}x}$.	
\item*  $\pi \int\limits_{0}^{1}{{{e}^{8x}}\text{d}x}$.	
\item  $\pi \int\limits_{0}^{1}{{{e}^{4x}}\text{d}x}$.	
\item  $\int\limits_{0}^{1}{{{e}^{8x}}\text{d}x}$.
\end{multi}

\begin{multi}[points=1]{Câu 35}
 Số giao điểm của đồ thị hàm số $y=-{{x}^{3}}+7x$với trục hoành là

\item  $0$.	
\item*  $3$.	
\item  $2$.	
\item  $1$.
\end{multi}

\begin{multi}[points=1]{Câu 36}
 Tập nghiệm của bất phương trình ${{\log }_{3}}\left( 13-{{x}^{2}} \right)\ge 2$ là

\item  $\left( -\infty ;-2 \right]\cup \left[ 2:+\infty  \right)$.	
\item  $\left( -\infty ;2 \right]$.	
\item  $\left( 0;2 \right]$.	
\item*  $\left[ -2;2 \right]$.
\end{multi}

\begin{multi}[points=1]{Câu 37}
 Biết $\int\limits_{0}^{1}{\left[ f\left( x \right)+2x \right]}dx=3$. Khi đó $\int\limits_{0}^{1}{f\left( x \right)\text{d}x}$ bằng

\item  $1$.	
\item  $5$.	
\item  $3$.	
\item*  $2$.
\end{multi}

\begin{multi}[points=1]{Câu 38}
 Trong không gian $Oxyz$, cho $M\left( 1;2;-3 \right)$ và mặt phẳng \[(P):\text{ 2}x-y+3z-1=0\]. Phương trình của đường thẳng đi qua điểm $M$ và vuông góc với $(P)$ là

\item  $\left\{  & x=2+t \\  & y=-1+2t \\  & z=3-3t \\  \right.$.	
\item  $\left\{   & x=-1+2t \\  & y=-2-t \\  & z=3+3t \\  \right.$.	
\item*  $\left\{   & x=1+2t \\  & y=2-t \\  & z=-3+3t \\  \right.$.	
\item  $\left\{   & x=1-2t \\  & y=2-t \\  & z=-3-3t \\  \right.$.
\end{multi}

\begin{multi}[points=1]{Câu 39}
 Năm \[2020\]một hãng xe niêm yết giá bán loại xe X là \$750.000.000\$ đồng và dự định trong \$10\$ năm tiếp theo, mỗi năm giảm \$2\%\$ giá bán so với giá bán của năm liền trước. Theo dự định đó năm \$2025\$ hãng xe ô tôt niêm yết giá bán loại xe X là bao nhiêu ( kết quả làm tròn đến hàng nghìn ) 

\item*  $677.941.000$ đồng.		
\item  $675.000.000$ đồng.	
\item  $664.382.000$ đồng.		
\item  $691.776.000$ đồng.
\end{multi}

\begin{multi}[points=1]{Câu 40}
 Biết $F\left( x \right)={{e}^{x}}-2{{x}^{2}}$ là một nguyên hàm của hàm số $f\left( x \right)$ trên $\mathbb{R}$. Khi đó $\int{f\left( 2x \right)}\,dx$ bằng

\item  $2{{e}^{x}}-4{{x}^{2}}+C$.	
\item*  $\frac{1}{2}{{e}^{2x}}-4{{x}^{2}}+C$.	
\item  ${{e}^{2x}}-8{{x}^{2}}+C$.	
\item  $\frac{1}{2}{{e}^{2x}}-2{{x}^{2}}+C$.
\end{multi}

\begin{multi}[points=1]{Câu 41}
 Cho hình nón $\left( N \right)$ có đỉnh $S$, bán kính đáy bằng $\sqrt{3}a$ và độ dài đường sinh bằng $4a$. Gọi $\left( T \right)$ là mặt cầu đi qua $S$ và đường tròn đáy của $\left( N \right)$. Bán kính của $\left( T \right)$ bằng

\item  $\frac{2\sqrt{10}a}{3}$.	
\item  $\frac{16\sqrt{13}a}{13}$.	
\item*  $\frac{8\sqrt{13}a}{13}$.	
\item  $\sqrt{13}a$.
\end{multi}

\begin{multi}[points=1]{Câu 42}
 Tập hợp tất cả các giá trị thức của tham số $m$ để hàm số $y={{x}^{3}}-3{{x}^{2}}+\left( 5-m \right)x$ đồng biến trên khoảng $\left( 2;+\infty  \right)$ là

\item  $\left( -\infty ;2 \right)$.	
\item  $\left( -\infty ;5 \right)$.	
\item*  $\left( -\infty ;5 \right]$.	
\item  $\left( -\infty ;2 \right]$.
\end{multi}

\begin{multi}[points=1]{Câu 43}
 Gọi $S$là tập hợp tất cả các số tự nhiên có 6 chữ số đôi một khác nhau. Chọn ngẫu nhiên một số thuộc $S$, xác suất để số đó có hai chữ số tận cùng có cùng tính chẵn lẻ bằng

\item*  $\frac{4}{9}$.	
\item  $\frac{2}{9}$.	
\item  $\frac{2}{5}$.	
\item  $\frac{1}{3}$.
\end{multi}

\begin{multi}[points=1]{Câu 44}
 Xét các số thực thỏa mãn \[{{2}^{{{x}^{2}}+{{y}^{2}}+1}}\le \left( {{x}^{2}}+{{y}^{2}}-2x+2 \right){{4}^{x}}\]. Giá trị lớn nhất của biểu thức \[P=\frac{8x+4}{2x-y+1}\] gần với giá trị nào sau đây nhất? 

\item  $9$	
\item  $6$.	
\item*  $7$.	
\item  $8$.
\end{multi}

\begin{multi}[points=1]{Câu 45}
 Cho hình chóp đều $S.ABCD$ có cạnh đáy bằng $4a$, cạnh bên bằng $2\sqrt{3}a$ và $O$ là tâm của đáy. Gọi $M$, $N$, $P$, $Q$ lần lượt là hình chiếu vuông góc của $O$ lên các mặt phẳng $(SAB)$, $(SBC)$, $(SCD)$và $(SDA)$. Thể tích của khối chóp $O.MNPQ$ bằng

\item  $\frac{4{{a}^{3}}}{3}$.	
\item  $\frac{64{{a}^{3}}}{81}$.	
\item  $\frac{128{{a}^{3}}}{81}$.	
\item*  $\frac{2{{a}^{3}}}{3}$.
\end{multi}

\begin{multi}[points=1]{Câu 46}
 Cho hình chóp $S.ABC$ có đáy là tam giác $ABC$ vuông cân tại $A$, $AB=a$, $SA$ vuông góc với mặt phẳng đáy, $SA=2a$, M là trung điểm của $BC$. Khoảng cách giữa $AC$ và $SM$ là

\item  $\frac{a}{2}$.	
\item  $\frac{a\sqrt{2}}{2}$.	
\item*  $\frac{2a\sqrt{17}}{17}$.	
\item  $\frac{2a}{3}$
\end{multi}

\begin{multi}[points=1]{Câu 47}
 Cho hàm số $f\left( x \right)=a{{x}^{3}}+b{{x}^{2}}+cx+d\text{  }\left( a,b,c,d\in \mathbb{R} \right)$ có bảng biến thiên như sau
Có bao nhiêu số dương trong các số \[a,\text{ }b,\text{ }c,\text{ }d\]?

\item  $2$.	
\item  $4$.	
\item  $1$.	
\item*  $3$.
\end{multi}

\begin{multi}[points=1]{Câu 48}
 Cho hàm số $f\left( x \right)$ có $f\left( 0 \right)=0$. Biết $y={f}'\left( x \right)$ là hàm số bậc bốn và có đồ thị là đường cong trong hình bên. Số điểm cực trị của hàm số $g\left( x \right)=\left| f\left( {{x}^{3}} \right)+x \right|$ là

\item  4.	
\item*  5.	
\item  3.	
\item  6.
\end{multi}

\begin{multi}[points=1]{Câu 49}
 Có bao nhiêu cặp số nguyên dương \[\left( m,n \right)\] sao cho \[m+n\le 16\] và ứng với mỗi cặp \[\left( m,n \right)\] tồn tại đúng \[3\] số thực \[a\in \left( -1;1 \right)\] thỏa mãn \[2{{a}^{m}}-n\ln \left( a+\sqrt{{{a}^{2}}+1} \right)\]? 

\item  \[16\].	
\item  \[14\].	
\item*  15.	
\item  \[13\].
\end{multi}

\begin{multi}[points=1]{Câu 50}
 Cho hàm số $y=f\left( x \right)$ có bảng biến thiên như hình vẽ:Có bao nhiêu giá trị nguyên của tham số $m$ để phương trình $6f\left( {{x}^{2}}-4x \right)=m$ có ít nhất ba nghiệm thực phân biệt thuộc khoảng $\left( 0\,;\,+\infty  \right)$? 

\item  25.	
\item  30.	
\item  29.	
\item*  24.
\end{multi}

\end{quiz}

\end{document}