%!TEX xelatex
\documentclass{book}
\usepackage{moodle}
\usepackage{graphicx}
\usepackage{amssymb}
\usepackage{amsmath}
\begin{document}


\begin{quiz}{Hóa 2021 - 357}

\begin{multi}[points=1]{Câu 1}
 Kim loại nhôm, sắt, crom bị thụ động hóa trong dung dịch nào?

\item*  $H_2SO_4$ đặc nguội.	
\item  KOH.	
\item  $H_2SO_4$ loãng.	
\item  NaOH.
\end{multi}

\begin{multi}[points=1]{Câu 2}
 Cho sơ đồ phản ứng sau: Vậy X, Y lần lượt là: 

\item  $Al_2O_3, NaHCO_3.$	
\item*  $Al_2O_3, Al(OH)_3.$	
\item  $Al(OH)_3, Al_2O_3.$	
\item  $AlCl_3, Al(OH)_3.$
\end{multi}

\begin{multi}[points=1]{Câu 3}
 Cho dãy các chất : $FeO, Fe, Cr(OH)_3, Cr_2O_3$. Số chất trong dãy phản ứng được với dung dịch HCl là 

\item  1.	
\item*  4.	
\item  3.	
\item  2.
\end{multi}

\begin{multi}[points=1]{Câu 4}
 Chất X tác dụng với dung dịch HCl. Khi chất X tác dụng với dung dịch $Ca(OH)_2$ sinh ra kết tủa. Chất X là

\item  $CaCO_3.$	
\item  $BaCl_2.$	
\item  $AlCl_3.$	
\item*  $Ca(HCO3)_2.$
\end{multi}

\begin{multi}[points=1]{Câu 5}
 Dãy các chất : $Al, Al(OH)_3, Al_2O_3, AlCl_3$. Số chất lưỡng tính trong dãy là

\item  1.	
\item*  2.	
\item  4.	
\item  3.
\end{multi}

\begin{multi}[points=1]{Câu 6}
 Kim loại Fe tác dụng với dung dịch nào sau đây tạo thành muối sắt(III)? 

\item  $H_2SO_4$ loãng.	
\item  $HNO_3$ đặc, nguội.	
\item*  $HNO_3$ loãng dư.	
\item  dung dịch $CuSO_4$.
\end{multi}

\begin{multi}[points=1]{Câu 7}
 Cho dãy các kim loại kiềm: Na, Li, Cs, Rb. Kim loại có nhiệt độ nóng chảy thấp nhất là

\item*  Cs.	
\item  Li.	
\item  Rb.	
\item  Na.
\end{multi}

\begin{multi}[points=1]{Câu 8}
 Oxi hoá $NH_3$ bằng $CrO_3$ sinh ra $N_2, H_2O và Cr_2O_3$. Số phân tử $NH_3$ tác dụng với một phân tử $CrO_3$ là

\item  3.	
\item*  1.	
\item  4.	
\item  2.
\end{multi}

\begin{multi}[points=1]{Câu 9}
 Chất thuộc loại cacbohiđrat là 

\item  protein.	
\item  poli(vinylclorua).	
\item  glixerol.	
\item*  xenlulozơ.
\end{multi}

\begin{multi}[points=1]{Câu 10}
 Cho dãy các chất: $H_2NCH(CH_3)COOH, C_6H_5OH (phenol), CH_3COOC_2H_5, C_2H_5OH, CH_3NH_3Cl$. Số chất trong dãy phản ứng với dung dịch KOH đun nóng là

\item*  4.	
\item  5.	
\item  3.	
\item  2.
\end{multi}

\begin{multi}[points=1]{Câu 11}
 Chất tham gia phản ứng trùng ngưng là

\item*  $H_2NCH_2COOH$.	
\item  $C_2H_5OH$.	
\item  $CH_3COOH$.	
\item  $CH_2=CH-COOH$.
\end{multi}

\begin{multi}[points=1]{Câu 12}
 Trong tự nhiên, caxi sunphat tồn tại dưới dạng muối ngậm nước $(CaSO4.2H2O)$ được gọi là

\item  Đá vôi.	
\item  Thạch cao nung.	
\item*  Thạch cao sống.	
\item  Thạch cao khan.
\end{multi}

\begin{multi}[points=1]{Câu 13}
 Trường hợp nào sau đây xảy ra ăn mòn điện hóa? 
Trường hợp xảy ra sự ăn mòn điện hóa là “Thanh kẽm nhúng trong dung dịch $CuSO_4”$. Ở đây, cặp điện cực là Zn – Cu, dung dịch chất điện li là $CuSO_4$. Các trường hợp còn lại, kim loại bị ăn mòn hóa học. 

\item*  Thanh kẽm nhúng trong dung dịch $CuSO_4$.	
\item  Đốt lá sắt trong khí $Cl_2$.	
\item  Thanh nhôm nhúng trong dung dịch $H_2SO_4$ loãng.	
\item  Sợi dây bạc nhúng trong dung dịch $HNO_3$.
\end{multi}

\begin{multi}[points=1]{Câu 14}
 Chất tác dụng với $Cu(OH)_2$ tạo sản phẩm có màu tím là

\item  anđehit axetic.	
\item  xenlulozơ.	
\item*  peptit.	
\item  tinh bột.
\end{multi}

\begin{multi}[points=1]{Câu 15}
 Nhận xét nào sau đây không đúng? 

\item  Metyl fomat có nhiệt độ sôi thấp hơn axit axetic.	
\item*  Metyl axetat là đồng phân của axit axetic.	
\item  Poli(metyl metacrylat) được dùng làm thủy tinh hữu cơ.	
\item  Các este thường nhẹ hơn nước và ít tan trong nước.
\end{multi}

\begin{multi}[points=1]{Câu 16}
 Để loại các khí: $SO_2; NO_2; HF$ trong khí thải công nghiệp, người ta thường dẫn khí thải đi qua dung dịch nào dưới đây? 

\item*  $Ca(OH)_2$.	
\item  HCl.	
\item  NaCl.	
\item  NaOH.
\end{multi}

\begin{multi}[points=1]{Câu 17}
 Điều chế kim loại K bằng cách

\item  Điện phân dung dịch KCl có màng ngăn.	
\item*  Điện phân KCl nóng chảy.	
\item  Dùng CO khử K+ trong $K_2O$ ở nhiệt độ cao.	
\item  Điện phân dung dịch KCl không có màng ngăn.
\end{multi}

\begin{multi}[points=1]{Câu 18}
 Cho dãy các chất: $Al_2O_3, KOH, Al(OH)_3, CaO$. Số chất trong dãy tác dụng với $H_2O$

\item  2.	
\item  4.	
\item*  1.	
\item  3.
\end{multi}

\begin{multi}[points=1]{Câu 19}
 Polime bị thuỷ phân cho $\alpha$-amino axit là

\item  polistiren.	
\item  polisaccarit.	
\item  nilon-6,6.	
\item*  polipeptit.
\end{multi}

\begin{multi}[points=1]{Câu 20}
 Chất nào sau đây tác dụng với dung dịch NaOH sinh ra glixerol? 

\item*  Triolein.	
\item  Glucozơ.	
\item  Saccarozơ.	
\item  Metyl axetat.
\end{multi}

\begin{multi}[points=1]{Câu 21}
 Các hiđroxit X, Y, Z, T có một số đặc điểm sau: 
X, Y, Z, T lần lượt là: 

\item  $Ba(OH)_2, Al(OH)_3, Fe(OH)_3, NaOH$.	
\item  $NaOH, Al(OH)_3, Fe(OH)_3, Ba(OH)_2$.	
\item  $Ba(OH)_2, Fe(OH)_3, Al(OH)_3, NaOH$.	
\item*  $NaOH, Fe(OH)_3, Al(OH)_3, Ba(OH)_2$.
\end{multi}

\begin{multi}[points=1]{Câu 22}
 Polime được điều chế bằng phản ứng trùng ngưng là? 

\item*  nilon-6,6.	
\item  poli(metylmetacrylat).	
\item  poli(vinylclorua).	
\item  polietilen.
\end{multi}

\begin{multi}[points=1]{Câu 23}
 Đốt cháy hoàn toàn 0,15 mol một este X, thu được 10,08 lít khí $CO_2$ (đktc) và 8,1 gam $H_2O$. Công thức phân tử của X là 

\item  $C_2H_4O_2$.	
\item*  $C_3H_6O_2$.	
\item  $C_5H_{10}O_2$.	
\item  $C_4H_8O_2$.
\end{multi}

\begin{multi}[points=1]{Câu 24}
 Hòa tan hoàn toàn hỗn hợp gồm 0,03 mol Cu và 0,09 mol Mg vào dung dịch chứa 0,07 mol $KNO_3$ và 0,16 mol $H_2SO_4$ loãng thì thu được dung dịch Y chỉ chứa các muối sunfat trung hòa và 1,12 lít (đktc) hỗn hợp khí X gồm các oxit của nitơ có tỉ khối so với $H_2$ là x. Giá trị của x là

\item  19,5.	
\item  20,1.	
\item  18,2.	
\item*  19,6.
\end{multi}

\begin{multi}[points=1]{Câu 25}
 Cho 0,1 mol este tạo bởi axit 2 lần axit hai chức và ancol một  ancol đơn chức tác dụng hoàn toàn với dung dịch NaOH, thu được 6,4 gam ancol và một lượng muối có khối lượng nhiều hơn 13,56\% khối lượng este. Công thức cấu tạo của este là 

\item  $C_2H_5OOC-COOC_2H_5$.	
\item  $C_2H_5OOC-COOCH_3$.	
\item  $CH_3OOC-CH_2-COOCH_3$.	
\item*  $CH_3OOC-COOCH_3$.
\end{multi}

\begin{multi}[points=1]{Câu 26}
 Khi thủy phân hoàn toàn một tetrapeptit X mạch hở chỉ thu được amino axit chứa 1 nhóm $–NH_2$ và 1 nhóm –COOH. Cho m gam X tác dụng vừa đủ với 0,3 mol NaOH thu được 34,95 gam muối. Giá trị của m là

\item  22,95.	
\item  21,15.	
\item*  24,30.	
\item  21,60.
\end{multi}

\begin{multi}[points=1]{Câu 27}
 Có các nhận định sau: (1) Sản phẩm của phản ứng giữa axit cacboxylic và ancol là este; (2) Este là hợp chất hữu cơ trong phân tử có nhóm $-COO^-$; (3) Este no, đơn chức, mạch hở có công thức phân tử $C_nH_{2n}O_2$, với n >= 2; (4) Hợp chất $CH_3COOC_2H_5$ thuộc loại este. Số nhận định đúng là

\item  2	
\item  4.	
\item*  3.	
\item  1
\end{multi}

\begin{multi}[points=1]{Câu 28}
 Cho luồng khí CO dư đi qua ống sứ đựng 5,36 gam hỗn hợp FeO và $Fe_2O_3$ (nung nóng), thu được m gam chất rắn và hỗn hợp khí X. Cho X vào dung dịch $Ca(OH)_2$ dư, thu được 9 gam kết tủa. Biết các phản ứng xảy ra hoàn toàn. Giá trị của m là

\item  3,75.	
\item*  3,92.	
\item  3,88.	
\item  2,48.
\end{multi}

\begin{multi}[points=1]{Câu 29}
 Khối lượng phân tử của tơ capron là 15000 đvC. Số mắt xích trung bình trong phân tử của loại tơ này gần nhất là

\item  145.	
\item 118
\item  113
\item*  133.
\end{multi}

\begin{multi}[points=1]{Câu 30}
 Cho 15 gam hỗn hợp các amin gồm anilin, metylamin, đimetylamin, đietylmetylamin tác dụng vừa đủ với 50 ml dung dịch HCl 1M. Khối lượng sản phẩm thu được là

\item  15,925 gam.	
\item  20,18 gam.	
\item  21,123 gam.	
\item*  16,825 gam.
\end{multi}

\begin{multi}[points=1]{Câu 31}
 Cho 115,3 gam hỗn hợp hai muối $MgCO_3$ và $RCO_3$ vào dung dịch $H_2SO_4$ loãng, thu được 4,48 lít khí $CO_2$ (đktc), chất rắn X và dung dịch Y chứa 12 gam muối. Nung X đến khối lượng không đổi, thu được chất rắn Z và 11,2 lít khí $CO_2$ (đktc). Khối lượng của Z là

\item  92,1 gam.	
\item  80,9 gam.	
\item*  88,5 gam.	
\item  84,5 gam.
\end{multi}

\begin{multi}[points=1]{Câu 32}
 Cho từ từ chất X vào dung dịch Y, sự biến thiên lượng kết tủa Z tạo thành trong thí nghiệm được biểu diễn trên đồ thị sau:         
Phát biểu sau đây đúng là: 

\item  X là khí $CO_2$; Y là dung dịch $Ca(OH)_2$; Z là $CaCO_3$.	
\item  X là dung dịch NaOH; Y là dung dịch $AlCl_3$; Z là $Al(OH)_3$.	
\item  X là dung dịch NaOH; Y là dung dịch gồm HCl và $AlCl_3$; Z là $Al(OH)_3$.	
\item*  X là khí $CO_2$; Y là dung dịch gồm NaOH và $Ca(OH)_2$; Z là $CaCO_3$.
\end{multi}

\begin{multi}[points=1]{Câu 33}
 Hòa tan 1,12 gam Fe bằng 300 ml dung dịch HCl 0,2M, thu được dung dịch X và khí $H_2$. Cho dung dịch $AgNO_3$ dư vào X, thu được khí NO (sản phẩm khử duy nhất của N+5) và m gam kết tủa. Biết các phản ứng xảy ra hoàn toàn. Giá trị của m là

\item  7,36.	
\item  10,23.	
\item*  9,15.	
\item  8,61.
\end{multi}

\begin{multi}[points=1]{Câu 34}
 Cho các loại hợp chất: amino axit (X), muối amoni của axit cacboxylic (Y), amin (Z), este của amino axit (T). Dãy gồm các loại hợp chất đều tác dụng được với dung dịch NaOH và đều tác dụng được với dung dịch HCl là: 

\item*  X, Y, T.	
\item  X, Y, Z.	
\item  X, Y, Z, T.	
\item  Y, Z, T.
\end{multi}

\begin{multi}[points=1]{Câu 35}
 Khi lên men 360 gam glucozơ với hiệu suất 100\%, khối lượng ancol etylic thu được là

\item*  184 gam.	
\item  276 gam.	
\item  92 gam.	
\item  138 gam.
\end{multi}

\begin{multi}[points=1]{Câu 36}
 Hoà tan 6,5 gam Zn trong dung dịch axit HCl dư, sau phản ứng cô cạn dung dịch thì số gam muối khan thu được là

\item  20,7gam.	
\item  27,2 gam.	
\item*  13,6 gam.	
\item  14,96gam.
\end{multi}

\begin{multi}[points=1]{Câu 37}
 Hòa tan hoàn toàn m gam Fe bằng dung dịch HNO3, thu được dung dịch X và 1,12 lít NO (đktc). Thêm dung dịch chứa 0,1 mol HCl vào dung dịch X thì thấy khí NO tiếp tục thoát ra và thu được dung dịch Y. Để phản ứng hết với các chất trong dung dịch Y cần 115 ml dung dịch NaOH 2M. Giá trị của m là

\item  3,36.	
\item*  3,92.	
\item  3,08.	
\item  2,8.
\end{multi}

\begin{multi}[points=1]{Câu 38}
 Cho m gam hỗn hợp M gồm 3 peptit X, Y, Z đều mạch hở và có tỉ lệ số mol nX : nY : nZ = 2 : 3 : 5. Thủy phân hoàn toàn N, thu được 60 gam Gly, 80,1 gam Ala, 117 gam Val. Biết số liên kết peptit trong X, Y, Z khác nhau và có tổng là 6. Giá trị của m là

\item  176,5.	
\item  257,1.	
\item*  226,5.	
\item  255,4.
\end{multi}

\begin{multi}[points=1]{Câu 39}
 Đốt a mol X là trieste của glixerol và axit đơn chức, mạch hở, thu được b mol $CO_2$ và c mol $H_2O$, biết b – c = 4a. Hiđro hóa m gam X cần 6,72 lít H2 (đktc), thu được 39 gam X’. Nếu cho m gam X phản ứng hoàn toàn với dung dịch chứa 0,7 mol NaOH, sau đó cô cạn dung dịch sau phản ứng thì thu được bao nhiêu gam chất rắn? 

\item  61,48 gam.	
\item  53,2 gam.	
\item  57,2 gam.	
\item*  52,6 gam.
\end{multi}

\begin{multi}[points=1]{Câu 40}
 Cho 30,8 gam hỗn hợp X gồm $Fe, FeO, FeCO_3, Mg, MgO và MgCO_3$ tác dụng vừa đủ với dung dịch $H_2SO_4$ loãng, thu được 7,84 lít (đktc) hỗn hợp khí Y gồm $CO_2, H_2$ và dung dịch Z chỉ chứa 60,4 gam hỗn hợp muối sunfat trung hòa. Tỉ khối của Y so với He là 6,5. Khối lượng của $MgSO_4$ có trong dung dịch Z là

\item  38,0 gam.	
\item  36,0 gam.	
\item*  30,0 gam.	
\item  33,6 gam.
\end{multi}

\end{quiz}

\end{document}