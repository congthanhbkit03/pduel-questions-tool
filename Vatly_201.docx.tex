%!TEX xelatex
\documentclass{book}
\usepackage{moodle}
\usepackage{graphicx}
\usepackage{amssymb}
\usepackage{amsmath}
\begin{document}


\begin{quiz}{2021 – Đề 201}

\begin{multi}[points=1]{Câu 1}
 Đặt một hiệu điện thế không đổi U vào hai đầu một đoạn mạch tiêu thụ điện năng thì cường độ dòng điện trong mạch là . Trong khoảng thời gian , điện năng tiêu thụ của đoạn mạch là . Công thức nào sau đây đúng ?

\item $~A=~\frac{U{{t}^{2}}}{I}.$	
\item $~A$ $=\frac{UI}{t}$.	
\item  $A=UI{{t}^{2}}$.	
\item*  $A=UIt$.
\end{multi}

\begin{multi}[points=1]{Câu 2}
 Sự phát quang của nhiều chất rắn có đặc điểm là ánh sáng phát quang có thể kéo dài một khoảng thời gian nào đó sau khi tắt ánh sáng kích thích. Sự phát quang này gọi là

\item*  sự lân quang.		
\item  sự nhiễu xạ ánh sáng.	
\item  sự tán sắc ánh sáng.		
\item  sự giao thoa ánh sáng.
\end{multi}

\begin{multi}[points=1]{Câu 3}
 Trong sự truyền sóng cơ, tần số dao động của một phần tử môi trường có sóng truyền qua được gọi là

\item  biên độ của sóng.	
\item  tốc độ truyền sóng.	
\item*  tần số của sóng.	
\item  năng lượng sóng.
\end{multi}

\begin{multi}[points=1]{Câu 4}
 Đặt điện áp xoay chiều có tần số góc $\omega $ vào hai đầu đoạn mạch gồm điện trở  mắc nối tiếp với cuộn cảm thuần có độ tự cảm . Hệ số công suất của đoạn mạch bằng

\item  $\frac{R}{R~-~{{\omega }^{2}}L}$.	
\item*  $\frac{R}{\sqrt{{{R}^{2}}~+~{{\omega }^{2}}{{L}^{2}}}}$.	
\item  $\frac{R}{\sqrt{{{R}^{2}}~+~\omega {{L}^{2}}}}$.	
\item  $\frac{R}{R~+~\omega L}$.
\end{multi}

\begin{multi}[points=1]{Câu 5}
 Khi nói về sóng điện từ, phát biểu nào sau đây sai? 

\item  Sóng điện từ có thể bị phản xạ, khúc xạ như ánh sáng.		
\item  Sóng điện từ mang năng lượng.	
\item*  Sóng điện là sóng ngang.		
\item  Sóng điện từ không lan truyền được trong điện môi.
\end{multi}

\begin{multi}[points=1]{Câu 6}
 Một con lắc lò xo gồm lò xo nhẹ và vật nhỏ có khối lượng $m$, đang dao động điều hòa. Gọi $v$ là vận tốc của vật. Đại lượng ${{W}_{}}$ $=\frac{1}{2}$ $m{{v}^{2}}$ được gọi là

\item  lực ma sát.	
\item  động năng của con lắc.	
\item  thế năng của con lắc.	
\item*  lực kéo về.
\end{multi}

\begin{multi}[points=1]{Câu 7}
 Đặc trưng nào sau đây không phải là đặc trưng vật lí của âm? 

\item  Cường độ âm.	
\item*  Tần số âm.	
\item  Độ to của âm.	
\item  Mức cường độ âm.
\end{multi}

\begin{multi}[points=1]{Câu 8}
 Đặt điện áp xoay chiều có giá trị hiệu dụng  vào hai đầu đoạn mạch chỉ có điện trở . Biểu thức cường độ dòng điện trong mạch có biểu thức $\text{i}=I\sqrt{2}\text{cos }\!\!~\!\!\text{ }t$ ($I>0$). Biểu thức điện áp giữa hai đầu đoạn mạch là 

\item  $u=U\sqrt{2}\text{cos}\left( \omega t-\frac{\pi }{2} \right)$.	
\item*  $u=U\sqrt{2}\cos \omega t$.	
\item  $u=U\cos \omega t$.		
\item  $u=U\sqrt{2}\text{cos}\left( \omega t+\frac{\pi }{2} \right)$.
\end{multi}

\begin{multi}[points=1]{Câu 9}
 Hiện tượng nào sau đây được ứng dụng để luyện nhôm ? 

\item  Hiện tượng siêu dẫn.		
\item*  Hiện tượng điện phân.	
\item  Hiện tượng nhiệt điện.	
\item  Hiện tượng đoản mạch.
\end{multi}

\begin{multi}[points=1]{Câu 10}
 Hai dao động điều hòa cùng phương có phương trình lần lượt là ${{x}_{1}}=~{{A}_{1}}\text{cos}\left( \omega t+{{\varphi }_{1}} \right)$ và ${{x}_{2}}=~{{A}_{2}}\text{cos}\left( \omega t+\varphi {{~}_{2}} \right)$ với  và $\omega $ là các hằng số dương. Dao động tổng hợp của hai dao động trên có biên độ là 
\item  Công thức nào sau đây đúng?
A.$~{{A}^{2}}=~A_{1}^{2}+\text{A}_{2}^{2}+2{{\text{A}}_{1}}{{\text{A}}_{2}}\text{cos}\left( {{\varphi }_{2}}+{{\varphi }_{1}} \right)$.	
\item $~{{A}^{2}}=~A_{1}^{2}-\text{A}_{2}^{2}+2{{\text{A}}_{1}}{{\text{A}}_{2}}\text{cos}\left( {{\varphi }_{2}}+{{\varphi }_{1}} \right)$.	
\item*  ${{A}^{2}}=~A_{1}^{2}+\text{A}_{2}^{2}+2{{\text{A}}_{1}}{{\text{A}}_{2}}\text{cos}\left( {{\varphi }_{2}}-{{\varphi }_{1}} \right)$.	
\item  ${{A}^{2}}=~A_{1}^{2}+\text{A}_{2}^{2}-2{{\text{A}}_{1}}{{\text{A}}_{2}}\text{cos}\left( {{\varphi }_{1}}+{{\varphi }_{2}} \right)$.
\end{multi}

\begin{multi}[points=1]{Câu 11}
 Hạt nhân ${}_{92}^{235}U$“bắt” một nơtron rồi vỡ thành hai mảnh nhẹ hơn và kèm theo vài nơtron. Đây là

\item  phản ứng nhiệt hạch.	
\item  hiện tượng quang điện.	
\item*  phản ứng phân hạch.		
\item  hiện tượng phóng xạ.
\end{multi}

\begin{multi}[points=1]{Câu 12}
 Trong chân không, ánh sáng màu tím có bước sóng nằm trong khoảng

\item  Từ 380 pm đến 440 pm.	
\item  Từ 380 pm đến 440 pm.	
\item*  Từ 380 nm đến 440 nm.	
\item  Từ 380 cm đến 440 cm.
\end{multi}

\begin{multi}[points=1]{Câu 13}
 Bộ phận nào sau đây là một trong ba bộ phận chính của máy quang phổ lăng kính? 

\item  Mạch tách sóng.	
\item  Pin quang điện.	
\item  Mạch biến điệu.	
\item*  Hệ tán sắc.
\end{multi}

\begin{multi}[points=1]{Câu 14}
 Khi một con lắc lò xo đang dao động tắt dần do tác dụng của lực ma sát thì cơ năng của con lắc chuyển hóa dần dần thành

\item  điện năng.	
\item  quang năng.	
\item  hóa năng.	
\item*  nhiệt năng.
\end{multi}

\begin{multi}[points=1]{Câu 15}
 Tia nào sau đây thường được sử dụng trong các bộ điều khiển từ xa để điều khiển hoạt động của tivi, quạt điện, máy điều hòa nhiệt độ? 

\item  Tia X.	
\item*  Tia hồng ngoại.	
\item  Tia $\gamma $.	
\item  Tia tử ngoại.
\end{multi}

\begin{multi}[points=1]{Câu 16}
 Máy phát điện xoay chiều một pha được cấu tạo bởi hai bộ phận chính là 

\item  cuộn sơ cấp và phần ứng.	
\item*  phần cảm và phần ứng.	
\item  cuộn thứ cấp và phần cảm.	
\item  cuộn sơ cấp và cuộn thứ cấp.
\end{multi}

\begin{multi}[points=1]{Câu 17}
 Dùng thuyết lượng tử ánh sáng có thể giải thích được

\item*  định luật về giới hạn quang điện. 	
\item  định luật phóng xạ.	
\item  hiện tượng giao thoa ánh sáng. 	
\item  hiện tượng nhiễu xạ ánh sáng. 
\end{multi}

\begin{multi}[points=1]{Câu 18}
 Đặt điện áp xoay chiều có tần số góc $\omega $ vào hai đầu đoạn mạch mắc nối tiếp gồm điện trở, cuộn cảm thuần có độ tự cảm $L$ và tụ điện có điện dung $C$. Điều kiện để trong mạch có cộng hưởng điện là

\item  $\omega LC=1$.	
\item  $2{{\omega }^{2}}LC=1$	
\item*  ${{\omega }^{2}}LC=1$.	
\item  $2\omega LC=1$.
\end{multi}

\begin{multi}[points=1]{Câu 19}
 Một con lắc đơn có chiều dài $\ell $, đang dao động điều hòa ở nơi có gia tốc trọng trường $g$. Đại lượng $T=2\pi \sqrt{\frac{l}{g}}$ được gọi là 

\item*  chu kì của dao động. 	
\item  tần số của dao động.	
\item  tần số góc của dao động. 	
\item  pha ban đầu của dao động. 
\end{multi}

\begin{multi}[points=1]{Câu 20}
 Theo thuyết tương đối, một vật đứng yên có năng lượng nghỉ ${{E}_{0}}$. Khi vật chuyển động thì có năng lượng toàn phần là $E$, động năng của vật lúc này là

\item $~{{W}_{}}$ $=\frac{1}{2}\left( E-{{E}_{0}} \right)$.	
\item  ${{W}_{}}=E+{{E}_{0}}$.	
\item*  ${{W}_{}}=E-{{E}_{0}}$.	
\item  ${{W}_{}}$ $=\frac{1}{2}\left( E-{{E}_{0}} \right)$.
\end{multi}

\begin{multi}[points=1]{Câu 21}
 Trong hệ $SI$, đơn vị của điện tích là

\item  vôn trên mét (V/m).	
\item*  culông (C).	
\item  fara (F).	
\item  vôn (V).
\end{multi}

\begin{multi}[points=1]{Câu 22}
 Trên một sợi dây đang có sóng dừng. Sóng truyền trên dây có bước sóng $\lambda $. Khoảng cách giữa hai bụng sóng liên tiếp là 

\item  $2\lambda $.	
\item  $\frac{\lambda }{4}$.	
\item  $\lambda $.	
\item*  $\frac{\lambda }{2}$.
\end{multi}

\begin{multi}[points=1]{Câu 23}
 Trong thí nghiệm giao thoa sóng ở mặt chất lỏng, tại hai điểm ${{S}_{1}}$ và ${{S}_{2}}$ có hai nguồn dao động cùng pha theo phương thẳng đứng, phát ra hai sóng kết hợp có bước sóng $1,2~\text{cm}$. Trên đoạn thẳng ${{S}_{1}}{{S}_{2}}$, khoảng cách giữa hai cực tiểu giao thoa liên tiếp bằng

\item  $0,3$.  cm.	
\item  $1,2$ cm.	
\item*  $0,6$ cm.	
\item  $2,4$ cm.
\end{multi}

\begin{multi}[points=1]{Câu 24}
 Một mạch chọn sóng ở một máy thu thanh là mạch dao động gồm cuộn cảm và tụ điện có điện dung $C$ thay đổi được. Biết rằng, muốn thu được sóng điện từ thì tần số riêng của mạch dao động phải bằng tần số của sóng điện từ cần thu (để có cộng hưởng). Khi $C={{C}_{0}}$ thì bước sóng của sóng điện từ mà máy này thu được là ${{\lambda }_{0}}$. Khi $C=9{{C}_{0}}$ thì bước sóng của sóng điện từ mà máy này thu được là

\item*  $3{{\lambda }_{0}}$.	
\item  $9{{\lambda }_{0}}$.	
\item  $\frac{{{\lambda }_{0}}}{9}$.	
\item  $\frac{{{\lambda }_{0}}}{3}$.
\end{multi}

\begin{multi}[points=1]{Câu 25}
 Trong thí nghiệm Y-âng về giao thoa ánh sáng, nguồn sáng phát sáng phát ra ánh sáng đơn sắc có bước sóng $\lambda $. Trên màn quan sát, vân sáng bậc 5 xuất hiện tại vị trí có hiệu đường đi của ánh sáng từ hai khe đến đó bằng

\item*  $5\lambda $.	
\item  $5,5\lambda $.	
\item  $4,5\lambda $.	
\item  $4\lambda $.
\end{multi}

\begin{multi}[points=1]{Câu 26}
 Cho một vòng dây dẫn kín dịch chuyển lại gần một nam châm thì trong vòng dây xuất hi một suất điện động cảm ứng. Đây là hiện tượng cảm ứng điện từ. Bản chất của hiện tượng cảm ứng điện từ này là quá trình chuyển hóa

\item*  cơ năng thành điện năng.	
\item  điện năng thành quang năng.	
\item  cơ năng thành quang năng.	
\item  điện năng thành hóa năng.
\end{multi}

\begin{multi}[points=1]{Câu 27}
 Một chất điểm dao động điều hòa với phương trình $x=8\text{cos}5t$ $\left( cm \right)$ ($t$ tính bằng $s$). Tốc độ của chất điểm khi đi qua vị trí cân bằng là

\item  $20$ cm/s.	
\item*  $40$ cm/s.	
\item  $200$ cm/s.	
\item  $100$ cm/s.
\end{multi}

\begin{multi}[points=1]{Câu 28}
 Cho phản ứng hạt nhân ${}_{1}^{2}\text{H }\!\!~\!\!\text{ }+\text{ }\!\!~\!\!\text{ }_{\text{Z}}^{\text{A}}\text{X}\to \text{ }\!\!~\!\!\text{ }_{3}^{6}\text{Li }\!\!~\!\!\text{ }+\text{ }\!\!~\!\!\text{ }_{5}^{10}\text{B}$. Giá trị của $\text{Z}$ là

\item*  7.	
\item  14.	
\item  9.	
\item  18.
\end{multi}

\begin{multi}[points=1]{Câu 29}
 Một máy biến áp lí tưởng có số vòng dây của cuộn sơ cấp và số vòng dây của cuộn thứ cấp lần lượt là ${{N}_{1}}=1100$ $\text{v }\!\!\grave{\mathrm{o}}\!\!\text{ ng}$ và ${{N}_{2}}$. Đặt điện áp xoay chiều có giá trị hiệu dụng $220~V$ vào hai đầu cuộn sơ cấp thì điện áp hiệu dụng giữa hai đầu cuộn thứ cấp để hở là $6~V$. Giá trị của ${{N}_{2}}~$là

\item  120 vòng.	
\item  60 vòng.	
\item  300 vòng.	
\item*  30 vòng.
\end{multi}

\begin{multi}[points=1]{Câu 30}
 Xét nguyên tử Hiđrô theo mẫu nguyên tử $Bo$. Khi nguyên tử chuyển từ trạng thái dừng có năng lượng $-0,85~eV$ sang trạng thái có năng lượng $-1,51~eV$ thì nó phát ra một phôtôn có năng lượng là

\item  $2,36~eV$.	
\item  $0,66~eV$.	
\item  $0,85~eV$.	
\item*  $1,51~eV$.
\end{multi}

\begin{multi}[points=1]{Câu 31}
 Đặt điện áp  $u=60\sqrt{2}\text{cos}\left( 300t+\frac{\pi }{3} \right)$ vào hai đầu mạch AB như hình bên, trong đó $R=170~\text{ }\!\!\Omega\!\!\text{ }$ và điện dung C của tụ điện thay đổi được. Khi $C={{C}_{1}}$  thì điện tích của bản tụ điện nối vào $N$ là $~q=5\sqrt{2}{{.10}^{-4}}.\text{cos}\left( 300t+\frac{\pi }{6} \right)$. Trong các biểu thức, t tính bằng s. Khi $C={{C}_{2}}$ thì điện áp hiệu dụng giữa hai đầu R đạt giá trị cực đại, giá trị cực đại đó bằng

\item*  51 V.	
\item  36 V.	
\item  60 V.	
\item  26 V.
\end{multi}

\begin{multi}[points=1]{Câu 32}
 Một sợi dây AB dài 66 cm và đầu $A$ cố định, đầu $B$ tự do, đang có sóng dừng với 6 nút sóng (kể cả đầu $A$). Sóng truyền từ $A$ đến $B$ gọi là sóng tới và sóng truyền từ $B$ về $A$ gọi là sóng phản xạ. Tại điểm $M$ trên dây cách $A$ một đoạn 64,5 cm, sóng tới và sóng phản xạ lệch pha nhau

\item  $\frac{\pi }{10}$.	
\item  $\frac{3\pi }{10}$.	
\item  $\frac{\pi }{2}$.	
\item*  $\frac{\pi }{4}$. 
\end{multi}

\begin{multi}[points=1]{Câu 33}
 Để xác định điện dung $C$ của một tụ điện, một học sinh mắc nối tiếp tụ điện này với một điện trở 20 Ω rồi mắc hai đầu đoạn mạch vào một nguồn điện xoay chiều có tần số thay đổi được. Dùng dao động kí điện tử để hiển thị đồng thời đồ thị điện áp giữa hai đầu điện trở và điện áp giữa hai đầu tụ điện (các đường hình sin). Thay đổi tần số của nguồn điện đến khi độ cao của hai đường hình sin trên màn hình dao động kí bằng nhau như hình bên. Biết dao động kí đã được chỉnh thang đo sao cho ứng với mỗi ô vuông trên màn hình thì cạnh nằm ngang là 0,005 s và cạnh thẳng đứng là 5 V. Giá trị của $C$ là 

\item  $3,{{0.10}^{-5}}\text{ }\!\!~\!\!\text{ F}$.	
\item  ${{12.10}^{-5}}\text{ }\!\!~\!\!\text{ F}$.	
\item  $6,{{0.10}^{-5}}\text{ }\!\!~\!\!\text{ F}$.	
\item*  ${{24.10}^{-5}}\text{ }\!\!~\!\!\text{ F}$.
\end{multi}

\begin{multi}[points=1]{Câu 34}
 Trong thí nghiệm Y-âng về giao thoa ánh sáng, hai khe hẹp cách nhau $0,6~mm$ và cách màn quan sát $1,2~m$. Chiếu sáng các khe bằng ánh sáng đơn sắc có bước sóng $\lambda ~(380~nm<\lambda <760~nm)$. Trên màn, điểm M cách vân trung tâm $2,5~mm$ là vị trí của một vân tối. Giá trị của $\lambda $ gần nhất với giá trị nào sau đây? 

\item*  $505\text{ }\!\!~\!\!\text{ nm}$.	
\item  $425\text{ }\!\!~\!\!\text{ nm}$.	
\item  $575\text{ }\!\!~\!\!\text{ nm}$.	
\item  $475\text{ }\!\!~\!\!\text{ nm}$.
\end{multi}

\begin{multi}[points=1]{Câu 35}
 Dao động của một vật là tổng hợp của hai dao động điều hòa có li độ lần lượt là ${{x}_{1}}$ và ${{x}_{2}}$. Hình bên là đồ thị biểu diễn sự phụ thuộc của ${{x}_{1}}$ và ${{x}_{2}}$ theo thời gian $t$. Biết độ lớn lực kéo về tác dụng lên vật ở thời điểm $t=0,2~s$ là $0,4~N$. Động năng của vật ở thời điểm $t=0,4~s$ là

\item  $6,4$ mJ.	
\item*  $4,8$ mJ.	
\item  $11,2$ mJ.	
\item  $15,6$ mJ. 
\end{multi}

\begin{multi}[points=1]{Câu 36}
 Đặt điện áp xoay chiều có tần số góc $\omega $ vào hai đầu đoạn mạch AB như hình bên ($H1$). Hình $H2$ là đồ thị biểu diễn sự phụ thuộc của điện áp ${{\text{u}}_{\text{AB}}}$ giữa hai điểm A và B, và điện áp ${{\text{u}}_{\text{MN}}}$ giữa hai điểm M và N theo thời gian t. Biết $63RC\omega =16$ và $r=18.$ Công suất tiêu thụ của mạch AB là

\item*  $20~W$.	
\item  $22~W$.	
\item  $16~W$.	
\item  $18~W$. 
\end{multi}

\begin{multi}[points=1]{Câu 37}
 Một mẫu chất chứa  ${}^{60}Co$ là chất phóng xạ với chu kì bán rã 5,27 năm, được sử dụng trong điều trị ung thư. Gọi ${{\text{N}}_{\text{o}}}$ là số hạt nhân ${}^{60}Co$  của mẫu phân rã trong 1 phút khi nó mới được sản xuất. Mẫu được coi là hết “hạn sử dụng” khi số hạt nhân ${}^{60}Co$  của mẫu phân rã trong 1 phút nhỏ hơn $0,7{{\text{N}}_{\text{o}}}$. Nếu mẫu được sản xuất vào tuần đầu tiên của tháng 8 năm 2020 thì “hạn sử dụng” của nó đến

\item*  tháng 4 năm 2023.	
\item  tháng 4 năm 2022.	
\item  tháng 6 năm 2024.	
\item  tháng 6 năm 2023.
\end{multi}

\begin{multi}[points=1]{Câu 38}
 Trong thí nghiệm giao thoa sóng ở mặt nước, hai nguồn kết hợp đặt tại hai điểm A và B, dao động cùng pha theo phương thẳng đứng, phát ra hai sóng lan truyền trên mặt nước với bước sóng . Ở mặt nước, C và D là hai điểm sao cho ABCD là hình vuông. Trên cạnh BC có $6$ điểm cực đại giao thoa và $7$ điểm cực tiểu giao thoa, trong đó P là điểm cực đại giao thoa gần B nhất và Q là điểm cực đại giao thoa gần C nhất. Khoảng cách xa nhất có thể giữa hai điểm P và Q là

\item  $8,04.$	
\item  $9,96.$	
\item*  $10,5.$	
\item  $8,93.$
\end{multi}

\begin{multi}[points=1]{Câu 39}
 Hai con lắc lò xo giống hệt nhau được gắn vào điểm $G$ của một giá cố định như hình bên. Trên phương nằm ngang và phương thẳng đứng, các con lắc đang dao động điều hòa với cùng biên độ $14~cm$, cùng chu kì $T$ nhưng vuông pha với nhau. Gọi ${{F}_{G}}$ là độ lớn hợp lực của các lực do hai lò xo tác dụng lên giá. Biết khoảng thời gian ngắn nhất giữa hai lần mà ${{F}_{G}}$ bằng trọng lượng của vật nhỏ của con lắc $\text{l }\!\!\grave{\mathrm{a}}\!\!\text{ }~\frac{T}{4}$. Lấy $g=10$m/s2. Giá trị của $T$ gần nhất với giá trị nào sau đây? 

\item  0,58 s.	
\item  0,69 s.	
\item*  0,74 s.	
\item  0,62 s.
\end{multi}

\begin{multi}[points=1]{Câu 40}
 Dùng mạch điện như hình bên để tạo ra dao động điện từ. Ban đầu khóa K vào chốt $a$, khi dòng điện qua nguồn điện ổn định thì chuyển khóa K đóng sang chốt $b$. Biết  E $=5~V;~r=1~;~R=2~;$ $L=\frac{9}{10\pi }~mH$  và $C$ $=\frac{1}{\pi }$ $\mu F$. Lấy $c=1,{{6.10}^{-19}}~C$. Trong khoảng thời gian $10\mu s$ kể từ thời điểm đóng K vào chốt $b$, có bao nhiêu electron đã chuyển đến bản tụ điện nối với khóa K? 

\item  4,48.1012 electron.		
\item  4,97.1012 electron.	
\item*  1,99.1012 electron.		
\item  1,79.1012 electron.
\end{multi}

\end{quiz}

\end{document}